\documentclass[]{article}
\usepackage{lmodern}
\usepackage{amssymb,amsmath}
\usepackage{ifxetex,ifluatex}
\usepackage{fixltx2e} % provides \textsubscript
\ifnum 0\ifxetex 1\fi\ifluatex 1\fi=0 % if pdftex
  \usepackage[T1]{fontenc}
  \usepackage[utf8]{inputenc}
\else % if luatex or xelatex
  \ifxetex
    \usepackage{mathspec}
  \else
    \usepackage{fontspec}
  \fi
  \defaultfontfeatures{Ligatures=TeX,Scale=MatchLowercase}
\fi
% use upquote if available, for straight quotes in verbatim environments
\IfFileExists{upquote.sty}{\usepackage{upquote}}{}
% use microtype if available
\IfFileExists{microtype.sty}{%
\usepackage{microtype}
\UseMicrotypeSet[protrusion]{basicmath} % disable protrusion for tt fonts
}{}
\usepackage[margin=1in]{geometry}
\usepackage{hyperref}
\hypersetup{unicode=true,
            pdftitle={Homework 9},
            pdfborder={0 0 0},
            breaklinks=true}
\urlstyle{same}  % don't use monospace font for urls
\usepackage{color}
\usepackage{fancyvrb}
\newcommand{\VerbBar}{|}
\newcommand{\VERB}{\Verb[commandchars=\\\{\}]}
\DefineVerbatimEnvironment{Highlighting}{Verbatim}{commandchars=\\\{\}}
% Add ',fontsize=\small' for more characters per line
\usepackage{framed}
\definecolor{shadecolor}{RGB}{248,248,248}
\newenvironment{Shaded}{\begin{snugshade}}{\end{snugshade}}
\newcommand{\KeywordTok}[1]{\textcolor[rgb]{0.13,0.29,0.53}{\textbf{#1}}}
\newcommand{\DataTypeTok}[1]{\textcolor[rgb]{0.13,0.29,0.53}{#1}}
\newcommand{\DecValTok}[1]{\textcolor[rgb]{0.00,0.00,0.81}{#1}}
\newcommand{\BaseNTok}[1]{\textcolor[rgb]{0.00,0.00,0.81}{#1}}
\newcommand{\FloatTok}[1]{\textcolor[rgb]{0.00,0.00,0.81}{#1}}
\newcommand{\ConstantTok}[1]{\textcolor[rgb]{0.00,0.00,0.00}{#1}}
\newcommand{\CharTok}[1]{\textcolor[rgb]{0.31,0.60,0.02}{#1}}
\newcommand{\SpecialCharTok}[1]{\textcolor[rgb]{0.00,0.00,0.00}{#1}}
\newcommand{\StringTok}[1]{\textcolor[rgb]{0.31,0.60,0.02}{#1}}
\newcommand{\VerbatimStringTok}[1]{\textcolor[rgb]{0.31,0.60,0.02}{#1}}
\newcommand{\SpecialStringTok}[1]{\textcolor[rgb]{0.31,0.60,0.02}{#1}}
\newcommand{\ImportTok}[1]{#1}
\newcommand{\CommentTok}[1]{\textcolor[rgb]{0.56,0.35,0.01}{\textit{#1}}}
\newcommand{\DocumentationTok}[1]{\textcolor[rgb]{0.56,0.35,0.01}{\textbf{\textit{#1}}}}
\newcommand{\AnnotationTok}[1]{\textcolor[rgb]{0.56,0.35,0.01}{\textbf{\textit{#1}}}}
\newcommand{\CommentVarTok}[1]{\textcolor[rgb]{0.56,0.35,0.01}{\textbf{\textit{#1}}}}
\newcommand{\OtherTok}[1]{\textcolor[rgb]{0.56,0.35,0.01}{#1}}
\newcommand{\FunctionTok}[1]{\textcolor[rgb]{0.00,0.00,0.00}{#1}}
\newcommand{\VariableTok}[1]{\textcolor[rgb]{0.00,0.00,0.00}{#1}}
\newcommand{\ControlFlowTok}[1]{\textcolor[rgb]{0.13,0.29,0.53}{\textbf{#1}}}
\newcommand{\OperatorTok}[1]{\textcolor[rgb]{0.81,0.36,0.00}{\textbf{#1}}}
\newcommand{\BuiltInTok}[1]{#1}
\newcommand{\ExtensionTok}[1]{#1}
\newcommand{\PreprocessorTok}[1]{\textcolor[rgb]{0.56,0.35,0.01}{\textit{#1}}}
\newcommand{\AttributeTok}[1]{\textcolor[rgb]{0.77,0.63,0.00}{#1}}
\newcommand{\RegionMarkerTok}[1]{#1}
\newcommand{\InformationTok}[1]{\textcolor[rgb]{0.56,0.35,0.01}{\textbf{\textit{#1}}}}
\newcommand{\WarningTok}[1]{\textcolor[rgb]{0.56,0.35,0.01}{\textbf{\textit{#1}}}}
\newcommand{\AlertTok}[1]{\textcolor[rgb]{0.94,0.16,0.16}{#1}}
\newcommand{\ErrorTok}[1]{\textcolor[rgb]{0.64,0.00,0.00}{\textbf{#1}}}
\newcommand{\NormalTok}[1]{#1}
\usepackage{graphicx,grffile}
\makeatletter
\def\maxwidth{\ifdim\Gin@nat@width>\linewidth\linewidth\else\Gin@nat@width\fi}
\def\maxheight{\ifdim\Gin@nat@height>\textheight\textheight\else\Gin@nat@height\fi}
\makeatother
% Scale images if necessary, so that they will not overflow the page
% margins by default, and it is still possible to overwrite the defaults
% using explicit options in \includegraphics[width, height, ...]{}
\setkeys{Gin}{width=\maxwidth,height=\maxheight,keepaspectratio}
\IfFileExists{parskip.sty}{%
\usepackage{parskip}
}{% else
\setlength{\parindent}{0pt}
\setlength{\parskip}{6pt plus 2pt minus 1pt}
}
\setlength{\emergencystretch}{3em}  % prevent overfull lines
\providecommand{\tightlist}{%
  \setlength{\itemsep}{0pt}\setlength{\parskip}{0pt}}
\setcounter{secnumdepth}{0}
% Redefines (sub)paragraphs to behave more like sections
\ifx\paragraph\undefined\else
\let\oldparagraph\paragraph
\renewcommand{\paragraph}[1]{\oldparagraph{#1}\mbox{}}
\fi
\ifx\subparagraph\undefined\else
\let\oldsubparagraph\subparagraph
\renewcommand{\subparagraph}[1]{\oldsubparagraph{#1}\mbox{}}
\fi

%%% Use protect on footnotes to avoid problems with footnotes in titles
\let\rmarkdownfootnote\footnote%
\def\footnote{\protect\rmarkdownfootnote}

%%% Change title format to be more compact
\usepackage{titling}

% Create subtitle command for use in maketitle
\newcommand{\subtitle}[1]{
  \posttitle{
    \begin{center}\large#1\end{center}
    }
}

\setlength{\droptitle}{-2em}

  \title{Homework 9}
    \pretitle{\vspace{\droptitle}\centering\huge}
  \posttitle{\par}
    \author{}
    \preauthor{}\postauthor{}
    \date{}
    \predate{}\postdate{}
  

\begin{document}
\maketitle

\subsection{12.1}\label{section}

As a tv research analyst, I am currently looking at whether or not
allowing compilations for our programs is beneficial to our online
viewership. While they allow for more content, they may potentially be
confusing viewers by seeing multiple programs with slightly different
program names (e.g. ``Show 1'' vs ``Show 1: Best Of'').

By testing various naming schemes, we can find out whether we gain
overall views, or even whether having compilations at all is detracting
from the total views.

\subsection{12.2}\label{section-1}

\begin{Shaded}
\begin{Highlighting}[]
\ControlFlowTok{if}\NormalTok{ (}\OperatorTok{!}\KeywordTok{require}\NormalTok{(}\StringTok{"FrF2"}\NormalTok{))}
\KeywordTok{install.packages}\NormalTok{(}\StringTok{'FrF2'}\NormalTok{, }\DataTypeTok{repos=}\StringTok{'http://cran.us.r-project.org'}\NormalTok{)}
\end{Highlighting}
\end{Shaded}

\begin{verbatim}
## Loading required package: FrF2
\end{verbatim}

\begin{verbatim}
## Loading required package: DoE.base
\end{verbatim}

\begin{verbatim}
## Loading required package: grid
\end{verbatim}

\begin{verbatim}
## Loading required package: conf.design
\end{verbatim}

\begin{verbatim}
## 
## Attaching package: 'DoE.base'
\end{verbatim}

\begin{verbatim}
## The following objects are masked from 'package:stats':
## 
##     aov, lm
\end{verbatim}

\begin{verbatim}
## The following object is masked from 'package:graphics':
## 
##     plot.design
\end{verbatim}

\begin{verbatim}
## The following object is masked from 'package:base':
## 
##     lengths
\end{verbatim}

\begin{Shaded}
\begin{Highlighting}[]
\KeywordTok{set.seed}\NormalTok{(}\DecValTok{42}\NormalTok{)}
\NormalTok{combo <-}\StringTok{ }\KeywordTok{FrF2}\NormalTok{(}\DecValTok{16}\NormalTok{,}\DecValTok{10}\NormalTok{)}
\NormalTok{combo}
\end{Highlighting}
\end{Shaded}

\begin{verbatim}
##     A  B  C  D  E  F  G  H  J  K
## 1  -1  1  1  1 -1 -1  1 -1  1 -1
## 2   1  1  1  1  1  1  1  1  1  1
## 3  -1 -1  1 -1  1 -1 -1  1  1 -1
## 4  -1  1 -1  1 -1  1 -1 -1 -1  1
## 5   1  1  1 -1  1  1  1 -1 -1 -1
## 6   1 -1  1 -1 -1  1 -1 -1  1  1
## 7   1  1 -1  1  1 -1 -1  1 -1 -1
## 8   1 -1 -1 -1 -1 -1  1 -1 -1 -1
## 9  -1 -1  1  1  1 -1 -1 -1 -1  1
## 10  1 -1  1  1 -1  1 -1  1 -1 -1
## 11 -1  1 -1 -1 -1  1 -1  1  1 -1
## 12  1  1 -1 -1  1 -1 -1 -1  1  1
## 13 -1  1  1 -1 -1 -1  1  1 -1  1
## 14 -1 -1 -1 -1  1  1  1  1 -1  1
## 15  1 -1 -1  1 -1 -1  1  1  1  1
## 16 -1 -1 -1  1  1  1  1 -1  1 -1
## class=design, type= FrF2
\end{verbatim}

The explanation is fairly simple - each of the factors from A through
are binary variables. We have 16 combinations of which factors to
keep.\\
If we look at the first combination, we keep \{B, C, G, H,K\}.\\
The combinations were chosen to never repeat between the 16, and also to
try to have the most variability in interactions. For example, both A
and B are included only in the 15th and 16th combinations. Considering
that there are 2\^{}10 = 1024 possible combinations, we have some
overlap with other variables.

\subsection{13.1}\label{section-2}

\subsubsection{a.) Binomial}\label{a.-binomial}

\paragraph{Whether or not a movie will profit - basically, we can think
of any binary question for the binomial distributions that a model such
as logistic regression may
solve.}\label{whether-or-not-a-movie-will-profit---basically-we-can-think-of-any-binary-question-for-the-binomial-distributions-that-a-model-such-as-logistic-regression-may-solve.}

\subsubsection{b.) Geometric}\label{b.-geometric}

\paragraph{At which day after release that the movie will break even -
this may never happen since the movie may never break even, just like
how a bat may never break during the experiment held by the class
lecture example. However, outside of that, we can try to see when a
specific production's movies tend to break even and find probability
parameters based off of
that.}\label{at-which-day-after-release-that-the-movie-will-break-even---this-may-never-happen-since-the-movie-may-never-break-even-just-like-how-a-bat-may-never-break-during-the-experiment-held-by-the-class-lecture-example.-however-outside-of-that-we-can-try-to-see-when-a-specific-productions-movies-tend-to-break-even-and-find-probability-parameters-based-off-of-that.}

\subsubsection{c.) Poisson}\label{c.-poisson}

\paragraph{The probability of a particular movie ticket (the golden
ticket) being
bought}\label{the-probability-of-a-particular-movie-ticket-the-golden-ticket-being-bought}

\subsubsection{d.) Exponential}\label{d.-exponential}

\paragraph{How many regular tickets are bought before each of the
several golden tickets being
found}\label{how-many-regular-tickets-are-bought-before-each-of-the-several-golden-tickets-being-found}

\subsubsection{e.) Weibull}\label{e.-weibull}

\paragraph{How long until a golden ticket is
bought}\label{how-long-until-a-golden-ticket-is-bought}


\end{document}
